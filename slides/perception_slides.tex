% Options for packages loaded elsewhere
\PassOptionsToPackage{unicode}{hyperref}
\PassOptionsToPackage{hyphens}{url}
%
\documentclass[
  12pt,
]{article}
\usepackage{amsmath,amssymb}
\usepackage{lmodern}
\usepackage{ifxetex,ifluatex}
\ifnum 0\ifxetex 1\fi\ifluatex 1\fi=0 % if pdftex
  \usepackage[T1]{fontenc}
  \usepackage[utf8]{inputenc}
  \usepackage{textcomp} % provide euro and other symbols
\else % if luatex or xetex
  \usepackage{unicode-math}
  \defaultfontfeatures{Scale=MatchLowercase}
  \defaultfontfeatures[\rmfamily]{Ligatures=TeX,Scale=1}
\fi
% Use upquote if available, for straight quotes in verbatim environments
\IfFileExists{upquote.sty}{\usepackage{upquote}}{}
\IfFileExists{microtype.sty}{% use microtype if available
  \usepackage[]{microtype}
  \UseMicrotypeSet[protrusion]{basicmath} % disable protrusion for tt fonts
}{}
\makeatletter
\@ifundefined{KOMAClassName}{% if non-KOMA class
  \IfFileExists{parskip.sty}{%
    \usepackage{parskip}
  }{% else
    \setlength{\parindent}{0pt}
    \setlength{\parskip}{6pt plus 2pt minus 1pt}}
}{% if KOMA class
  \KOMAoptions{parskip=half}}
\makeatother
\usepackage{xcolor}
\IfFileExists{xurl.sty}{\usepackage{xurl}}{} % add URL line breaks if available
\IfFileExists{bookmark.sty}{\usepackage{bookmark}}{\usepackage{hyperref}}
\hypersetup{
  pdftitle={The categorization of L3 French sounds at first exposure by Spanish-English bilinguals},
  hidelinks,
  pdfcreator={LaTeX via pandoc}}
\urlstyle{same} % disable monospaced font for URLs
\usepackage[margin=1in]{geometry}
\usepackage{graphicx}
\makeatletter
\def\maxwidth{\ifdim\Gin@nat@width>\linewidth\linewidth\else\Gin@nat@width\fi}
\def\maxheight{\ifdim\Gin@nat@height>\textheight\textheight\else\Gin@nat@height\fi}
\makeatother
% Scale images if necessary, so that they will not overflow the page
% margins by default, and it is still possible to overwrite the defaults
% using explicit options in \includegraphics[width, height, ...]{}
\setkeys{Gin}{width=\maxwidth,height=\maxheight,keepaspectratio}
% Set default figure placement to htbp
\makeatletter
\def\fps@figure{htbp}
\makeatother
\setlength{\emergencystretch}{3em} % prevent overfull lines
\providecommand{\tightlist}{%
  \setlength{\itemsep}{0pt}\setlength{\parskip}{0pt}}
\setcounter{secnumdepth}{-\maxdimen} % remove section numbering
\usepackage{tipa}
\usepackage{xcolor}
\ifluatex
  \usepackage{selnolig}  % disable illegal ligatures
\fi

\title{The categorization of L3 French sounds at first exposure by
Spanish-English bilinguals}
\author{}
\date{\vspace{-2.5em}10/26/21}

\begin{document}
\maketitle

The present study investigates bilinguals' categorization of L3 sounds
at first exposure in order to inform debates in L3 phonological
acquisition. Broadly, models of third language acquisition debate the
role of previously known languages in the acquisition of a third. In
particular, the starting point of third language acquisition has been
debated, in which some models suggest that one language holistically
influences the L3 (The Typological Primacy Model; Rothman, 2015, the L2
Status Factor, Bardel \& Falk, 2007), while others suggest that
property-by-property or gradient influence is possible (The Linguistic
Priximity Model; Westergaard et al., 2017, the Scalpel Model, 2017).

In the present study, participants who speak L1 American English and L2
Spanish heard a total of 20 French vowel sounds in 4 conditions (L1
bias, L2 bias, new sound, both languages) and chose the closest matching
L1 or L2 vowel sound given orthographically in language-specific carrier
words. This experimental paradigm has been used in previous studies
investigating the Perceptual Assimilation Model (Best \& Tyler, 2007),
and provides insight into how L2 sounds are categorized by naive
learners in order to evaluate the predictions of L2 models of speech
learning. Using and adapting this methodology to L3 acquisition, the
present study provides evidence that L3 sounds are categorized using
both L1 and L2 categories (top figure). Additionally, the same L3 sounds
were not always categorized the identically by the within subjects, nor
were all selections typically in one language (lower figure). These
results suggest that both languages of a bilingual are active during L3
perceptual routines and has implications for L3 models. The data best
support the Linguistic Proximity Model (Westergaard et al., 2017), while
providing counter-evidence for full transfer models of L3 acquisition
such as the L2 Status Factor and the Typological Primacy Model.

\newpage

\includegraphics[width=0.9\linewidth,height=0.8\textheight]{/Users/kyleparrish/Documents/GitHub/dissertation_parrish/docs/abstracts/new_sounds/figs/comb_results}

\hypertarget{references}{%
\subsection{References}\label{references}}

\begingroup
\setlength{\parindent}{-0.5in}
\setlength{\leftskip}{0.5in}
\phantom{.}

\textcolor{white}{\\} \vspace{-0.5in}

Bardel, C., \& Falk, Y. (2007). The role of the second language in third
language acquisition: The case of Germanic syntax. Second Language
Research, 23(4), 459--484.
\url{https://doi.org/10.1177/0267658307080557}

Best, C. T., \& Tyler, M. D. (2007). Nonnative and second-language
speech perception: Commonalities and complementarities. In O.-S. Bohn \&
M. J. Munro (Eds.), Language Learning \& Language Teaching (Vol. 17,
pp.~13--34). John Benjamins Publishing Company.
\url{https://doi.org/10.1075/lllt.17.07bes}

Rothman, J. (2015). Linguistic and cognitive motivations for the
Typological Primacy Model (TPM) of third language (L3) transfer: Timing
of acquisition and proficiency considered. Bilingualism: Language and
Cognition, 18(2), 179--190.
\url{https://doi.org/10.1017/S136672891300059X}

Slabakova, R. (2017). The scalpel model of third language acquisition.
International Journal of Bilingualism, 21(6), 651--665.
\url{https://doi.org/10.1177/1367006916655413}

Westergaard, M., Mitrofanova, N., Mykhaylyk, R., \& Rodina, Y. (2017).
Crosslinguistic influence in the acquisition of a third language: The
Linguistic Proximity Model. International Journal of Bilingualism,
21(6), 666--682. \url{https://doi.org/10.1177/1367006916648859}

\endgroup

\end{document}
