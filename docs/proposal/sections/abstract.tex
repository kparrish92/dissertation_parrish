The present dissertation aims to examine how bilinguals perceive and produce sounds in a language unfamilar to them in perception and production in order to uncover how a first and second language impact the acquistion of a third.
In particular, Spanish-English bilinguals will be recruited in both orders of acquistion (Spanish L1-English L2 and English L1-Spanish L2) and exposed to sounds in French (closer to Spanish) and German (closer to English).
Using this combination of languages, and by giving the same speakers two L3s to perceive and produce, allow for the relative impact of order of acquisition and cross-linguistic acoustic similarity to be examined in tandem. 
In order to elicit production in all 4 languages, a shadowing task in German and French, and a word reading task in Spanish and English will be carried out. 
For perception, a phoneme categorzation task and AX discrimination task will be done. 
Measures of voice-onset time and formant values will be evaluated in stops and vowels in all three languages in production, where perpcetion will use categorization of sounds and discrimination patterns to evaluate how cross-linguistic influence of the L1 and L2 impact L3 perception and production. 
The results of each study, as well as the production, perception interface, have implications for L3 models, which predict that whole language influence occurs (Rothman, 2015; Bardel and Falk, 2007) or that both languages are active during L3 acquistion (Westergaard et al., 2017; Slabakova, 2017).